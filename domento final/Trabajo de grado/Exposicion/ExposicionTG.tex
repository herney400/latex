%@AUTHOR: Cardel
%Configuracion del documento

\documentclass{beamer}
\usecolortheme[RGB={150,0,0}]{structure}
\usetheme[height=7mm]{Rochester}
\usepackage{graphicx}
\usepackage[utf8]{inputenc}
\usepackage[spanish]{babel}
\usepackage{ragged2e}
\usepackage{colortbl}
\usepackage{color}
\definecolor{naranja}{rgb}{1,0.5,0} % valores de las componentes roja, verde y azul (RGB)
\definecolor{rojo}{rgb}{1,0,0}
\definecolor{SteelBlue}{rgb}{0.3,0.5,0.7}


\author{Carlos Andr\'es Delgado S. Ing. } 
\title{Presentaci\'on Trabajo de Grado}
\subtitle{Diseño e implementación de una aplicaci\'on prototipo para la gestión del espectro radioel\'ectrico usando programaci\'on con restricciones}
\institute{Facultad de Ingeniería. Universidad del Valle}
\date{29 de Enero de 2013}
%Transparencia
\setbeamercovered{transparent}

%LOGO Univalle
\pgfdeclareimage[height=1.4cm]{logo}{imagenes/univalle}
\logo{\pgfuseimage{logo}}

%Para que en cada seccion aparezca la tabla de contenido
\AtBeginSection[]{
	\begin{frame}
	\frametitle{Contenido}
	\tableofcontents[currentsection]
\end{frame}
}

\begin{document}

	\begin{frame}
		\begin{tabular}{p{6.5cm}p{3cm}}
			\includegraphics[height=0.22\textheight]{imagenes/avispalogo.png} & \includegraphics[height=0.32\textheight]{imagenes/sisteluv.jpg} 	\\
		\end{tabular}
		\titlepage	 
		
	\end{frame}
	\begin{frame}
 		\frametitle{Contenido}
		\tableofcontents[hideallsections]
	\end{frame}
	
	%%Definición del problema
	\input{1.DefinicionProblema/definicion.tex}
	
	%%Objetivos
	\input{2.Objetivos/objetivos.tex}
	
	%%Solucion del problema
	\section{Solución del problema}
	
	\input{3.Solucion/modelo}
	Para realizar el proceso de implementación del proyecto se deben tomar en cuenta los pasos especificados en la metodología para aplicaciones Web del grupo Avispa \cite{Metodologia}.
\\\\
En este capítulo se dan las pautas más relevantes para la implementación de la aplicación que son:

\begin{itemize}
	\item Formatos de entradas y salidas.
	\item Parámetros de las aplicaciones.
\end{itemize}

Para facilitar la divulgación del proyecto, la aplicación se nombra cómo: \textbf{CaFeSA} que es la abreviación de \textit{Constraints Application For Enhanced Spectrum Allocation}, en español, aplicación por restricciones para asignación de espectro aumentado.

	\input{3.Solucion/aplicacion}
	\input{3.Solucion/otrosmetodos}
	
	%%Pruebas
	\input{4.Pruebas/pruebas}
	
	%%Conclusiones y trabajos futuros
	\input{5.Final/conclusiones}
	\input{5.Final/trabajosfuturos}

\end{document}
