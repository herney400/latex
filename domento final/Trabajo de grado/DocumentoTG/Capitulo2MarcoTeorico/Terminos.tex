\section{Glosario}
\begin{itemize}

\item{\textbf{Sistema de Información} Un sistema de información se define como un conjunto de componentes interrelacionados que recolectan (o recuperan), procesan, almacenan y distribuyen información para apoyar la toma de decisiones y el control de una organización:
	\begin{enumerate}
		\item \textbf{ Entrada de información:} Es el proceso mediante el cual el sistema toma los datos que requiere para procesar la información.		
		
		\item \textbf{ Almacenamiento de información: }El almacenamiento es una de las actividades o capacidades más importantes que tiene un computador, ya que a través de esta propiedad el sistema puede recordar la información guardada en la sección o proceso anterior
				
		\item \textbf{ Procesamiento de información:} Es la capacidad que tiene el sistema de información para efectuar cálculos de acuerdo con una secuencia de operaciones preestablecida.  			
	\end{enumerate} 
}


\item \textbf{Lenguaje UML:} Lenguaje Unificado de Modelado, es el lenguaje de modelado de sistemas de software más conocido y utilizado en la actualidad; está respaldado por el OMG (Object Management Group).  

\item \textbf{Base de Datos:}Uno de los objetivos de los sistemas de información es contar no solo con los recursos de información, sino también con los mecanismos para poder encontrar y recuperar estos recursos.  De esta manera, las bases de datos se han convertido en un elemento indispensable no solo para el funcionamiento de los grandes motores de búsqueda y la recuperfffción de información, sino también para la creación de sistemas de información en los que se precisa manejar grandes o pequeños volúmenes de información. 


\item \textbf{Web 2.0:} \cite{Cuadro}  La Web dos (punto) cero podría definirse como la promesa de una visión realizada: la Red convertida en un espacio social, con cabida para todos los agentes sociales, capaz de dar soporte y formar parte de una verdadera sociedad de la información, la comunicación y/o el conocimiento (Fumero y Roca, 2007).

\item \textbf{Minería de Datos:}La minería de datos es un campo de las ciencias de la computación referido al proceso que intenta descubrir patrones en grandes volúmenes de conjuntos de datos.
 

\end{itemize}
