
\section{Cálculo de costos}

\subsection{Número de cambios de asignación de canales para un operador}

Los cambios en los canales asignados a un operador $Cost_{1}$, se calcula de acuerdo a la ecuación \ref{equ:costoNb}

\begin{equation}
	\label{equ:costoNb}
	\begin{array}{cc}
		Cost_{1} = \lceil\frac{1}{2}*\sum \limits_{o \in OPi} \sum \limits^{C}_{c=1} CAO_{co}\rceil	
	\end{array}
\end{equation}

Se toma como el número de bloques la mitad del cálculo realizado en $CAO_{co}$ ya que este cuenta los limites de cada bloque, es decir que para un bloque se cuenta su inicio y su final, por lo que si hay $n$ bloques se contarán $2n$ en $CAO_{co}$; sin embargo en el caso de que el primer bloque inicia en la posición $1$ de la banda, no se cuenta, por lo que se toma el techo de este cálculo, ya que en ese caso da un número impar.

\subsubsection{Justificación}

El número de cambios en la asignación de un operador representa costos adicionales en la práctica ya que debe adquirir equipos adicionales para trabajar en canales que se encuentran separados. Por lo tanto, es ideal dar a un operador un bloque contiguo en la asignación para reducir costos.

\subsection{Diferencia entre el número de canales y el mayor bloque de canales libres}

La diferencia entre el número de canales de una banda y el tamaño del mayor bloque libre $Cost_{2}$ se calcula de acuerdo a la ecuación \ref{equ:cost2}.

\begin{equation}
	\label{equ:cost2}
	\begin{array}{cc}
		Cost_{2} = C - CLMmax
	\end{array}
\end{equation}

\subsubsection{Justificación}

Se busca que la asignación en una banda considere dejar un bloque de canales libres, para facilitar la asignación en futuros requerimientos, debido a que no se conoce cuantos canales se van a necesitar y se desea realizar la mejor asignación posible para esos requerimientos.

\subsection{Número de canales inutilizables}

Un canal libre es marcado como inutilizable cuando no puede ser asignado a ningún operador debido a la separación mínima exigida. En la ecuación \ref{equ:cost3} se muestra cómo calcular este costo.

\begin{equation}
	\label{equ:cost3}
	\begin{array}{cc}
			Cost_{3} = \sum \limits^{C}_{c=1} CAI_{c}
	\end{array}
\end{equation}

\subsubsection{Justificación}

En la asignación se debe buscar que el número de canales inutilizados por la separación sea la mínima posible debido a que esto representa un desperdicio de canales que no pueden ser asignados.

\subsection{Costo total}

El costo total se calcula a partir de la suma de pesos entre 1 y 100 especificados por el usuario conocidos como $peso_{1}, peso_{2}, peso_{3}$ multiplicado por cada uno de los costos.

\begin{equation}
	\label{equ:costR}
	\begin{array}{cc}
		Cost_{T} = peso_{1}*Cost_{1} + peso_{2}*Cost_{2} + peso_{3}*Cost_{3}
	\end{array}
\end{equation}
