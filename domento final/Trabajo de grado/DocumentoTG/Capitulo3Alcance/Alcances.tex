En este proyecto de grado se busca obtener una propuesta de solución al problema de gestión de los clientes que hacen uso de servicios de energía eléctrica usando agentes inteligentes basado en redes neuronales y minería de datos, con una aplicación prototipo en la que se puedan realizar desarrollos a futuro, por ello no se busca obtener un producto final. 


\section{Alcances metodológicos}	
El desarrollo de un prototipo software que permita modelar los procesos de información entre empresas distribuidoras de energía eléctrica y clientes para la construcción del precio de un \textbf{kwh}, en el marco de las tecnologías Smart Grid. En la aplicación prototipo se podrá tener un informe del consumo, precio y pagos históricos basado en información de la cual se obtendrá información relevante para para la toma de decisiones en la construcción del precio de un \textbf{Kwh}.

La gestión de información se elaborara únicamente utilizando técnicas de  minería de datos, entre las cuales se tiene en cuenta clientes industriales, residenciales.

En la practica se establecen normas que regulan el precio de la energía eléctrica, el cual depende de varios factores como la altitud, el estrato, el tipo de cliente (industrial, residencial),los cuales se han tenido en cuenta para el diseño del modelo que se implementara.
 
\section{Alcances prácticos}

Con respecto al proceso de modelar el proceso de información entre empresas distribuidoras de energía eléctrica y clientes se toma las siguientes consideraciones. 

\begin{itemize}
 
\item Se trabaja con datos históricos tanto para la aplicación de técnicas de minería de datos.
\item Las variables que intervienen en el precio de energía se ven reflejadas en el modelo de datos.
\item Los clientes se dividen en residenciales, industriales y comerciales, para los clientes residenciales se pueden se dividen por estrato.
\item Las zonas de la ciudad se divide en comunas, barrios. 
\item el día esta dividido en franjas horarias (madrugada, mañana, tarde y noche).

\end{itemize}

Para modelar la predicción del precio de energía eléctrica mediante agentes basado en redes neuronales se tiene en cuenta las siguientes alcances prácticos.

\begin{itemize}
 \item Se trabaja con datos históricos para el entrenamiento de las redes neuronales.
 \item Se trabaja con algoritmos de aprendizaje como \textbf{Backpropagation}.
 \item Las variables para modelar las entradas de la red neuronal son (clima, día ,altitud, estrato, franja horaria).
\end{itemize}
