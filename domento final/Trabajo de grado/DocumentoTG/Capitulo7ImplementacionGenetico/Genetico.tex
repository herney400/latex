\section{Diseño del algoritmo genético}

\subsection{El algoritmo} \label{sec:algGen}

El diseño del algoritmo genético tiene los siguientes pasos:

\begin{itemize}
	\item Definición del individuo.
	\item Función de aptitud.
	\item Criterio de selección.
	\item Criterio de cruce.
	\item Criterio de mutación.
	\item Selección de individuos para la próxima generación.
	\item Criterio de parada.
\end{itemize}

El objetivo del algoritmo es minimizar el costo de la solución.


\subsubsection{El individuo}

El individuo es una cadena de carácteres con ceros y unos de tamaño $N$ por $T=card(Opt)$, que representa la asignación final.

\subsubsection{Función de aptitud}

El costo total se define tomando en cuenta:

\begin{itemize}
	\item $Cost_{T}$ de la ecuación \ref{equ:costR}.
	\item $C$ número de canales.
	\item $NR_{e}$ número de restricciones que la solución infringe.
\end{itemize}

\begin{equation}
	\label{equ:costGR}
	\begin{array}{cc}
		CostGen_{T} =10000*T*C*NR_{e} + Cost_{T}
	\end{array}
\end{equation}

En la ecuación se busca darle un mayor en costo a una solución que presente infracciones a las restricciones del modelo, pero se presentan algunos problemas que se explican en la sección \ref{sec:genProblema}.

Para definir la función de aptitud se normalizan los costos de la población de acuerdo a una curva Gaussiana con con desviación estándar 1.5.
Se hace un corte de acuerdo al método de \textit{sigma truncation}\cite{Truncation}.\footnote{El método de sigma \textit{sigma truncation} consiste en descartar los individuos cuya función de aptitud sea peor que cierto valor de desviación estándar.}

\subsubsection{Criterio de selección}

La selección se hace por ruleta\footnote{La selección por ruleta utiliza un número aleatorio $r$ y el tamaño de la población inicial $P$ para generar $r_{j}=\frac{r+j-1}{j} \forall j=1,...P$}, para seleccionar un número de individuos igual al de la población inicial.

\subsubsection{Criterio de cruce}

Los individuos se cruzan con una probabilidad especificada por el usuario. Se utiliza el criterio de reemplazo por inserción \footnote{El reemplazo por inserción, significa tomar cierto porcentaje de características de cada individuo y mezclarlas para generar uno o más individuos hijos} para generar dos individuos hijos por pareja de individuos padres.

\subsubsection{Criterio de mutación}

Se selecciona un individuo de acuerdo a una probabilidad especificada por el usuario y se cambia el valor de uno de sus campos elegido con una función pseudoaleatoria, si el campo contiene un cero se cambia por uno y viceversa.

\subsubsection{Selección individuos próxima generación}

Se seleccionan los individuos generados con el criterio de cruce y un número de los mejores padres para completar el tamaño inicial de la población, así generación tras generación el tamaño de la población es estable.

\subsubsection{Criterio de parada}

Se realiza de acuerdo a un tiempo de ejecución especificado por el usuario.

\subsection{Problemas de modelado} \label{sec:genProblema}

Debido a la formula de la función de aptitud, el algoritmo teóricamente debe presentar el problema de mínimos locales, en este caso se presentaría en el momento que no se tenga infracciones a las restricciones y se tenga un costo que no es el óptimo, el algoritmo puede quedar atrapado allí debido a que para obtener un mejor costo debe pasar por un proceso que implica tener costos muy por encima del obtenido en ese punto.
\\\\
Para enfrentar ese problema el usuario puede decidir cuantas veces ejecutar el algoritmo para aumentar su precisión.

\subsection{Valores recomendados}

En la siguiente tabla se muestran los parámetros recomendados de acuerdo al método de solución propuesto \cite{4554416}.

\begin{center}
\begin{longtable}{|p{7cm}|p{7cm}|}
	\caption{Valores recomendados para la aplicación basada el algoritmos genéticos} \label{tabla:paramgenetico}\\
	\hline
	\cellcolor[gray]{0.9} \textbf{Parámetro} & \cellcolor[gray]{0.9}\textbf{Valor} \\
	\hline
	Número de individuos. & 20-40.\\
	\hline
	Probabilidad de cruce individuo seleccionado& 90\% (0.9).\\
	\hline
	Probabilidad mutación & 2\% (0.02).\\
	\hline
	Valor de sigma & 1.5 desviación estándar.\\
	\hline
\end{longtable}	
\end{center}

