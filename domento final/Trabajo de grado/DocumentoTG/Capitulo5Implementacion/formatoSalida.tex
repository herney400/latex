
\section{Formato de salidas} \label{sec:formatSal}

A continuación se definen la estructura de salidas que se va a utilizar en el aplicativo, un ejemplo puede ser consultado en el anexo salida de ejemplo ubicado en la página \pageref{ejemploSalida}.

\subsection{Formato general}

A partir de la metodología la estructura de la salida XML es la siguiente:

\begin{enumerate}
	\item{Raíz del documento:

	\lstset{frameround=fttt}
	\begin{lstlisting}[frame=trBL, language=XML]
	<solutions>
	</solutions>
	\end{lstlisting}}

	\item{Se establece un encabezado donde se encuentra la información de la ejecución de la aplicación:

	\lstset{frameround=fttt}
	\begin{lstlisting}[frame=trBL, language=XML]
	<head>
	</head>
	\end{lstlisting}}
	
	\item{Se estructura cada solución encontrada de la siguiente forma:

	\lstset{frameround=fttt}
	\begin{lstlisting}[frame=trBL, language=XML]
	<solution id="n">
	</solution>
	\end{lstlisting}}	
	
	$n$ indica el número de la solución, estas se encuentran ordenadas de acuerdo al mejor costo.

	\item{Se colocan los costos de cada solución y el reporte de la salida.

	\lstset{frameround=fttt}
	\begin{lstlisting}[frame=trBL, language=XML]
	<solution>
		<costs>
			...
		</costos>
		<report>
			...
		</report>
	</solution>
	\end{lstlisting}}
\end{enumerate}

Los campos de información de ejecución permiten evaluar el desempeño de la aplicación.

\begin{center}
\begin{longtable}{|p{7cm}|p{7cm}|}
	\caption{Campos de información de ejecución en general.}\\
	\hline
	\cellcolor[gray]{0.9} \textbf{Campo} & \cellcolor[gray]{0.9}\textbf{Descripción} \\
	\hline
	numSolutions & Número de soluciones encontradas.\\
	\hline
	memoryUsage & Uso del memoria del aplicativo.\\
	\hline
	executionTime & Tiempo de ejecución.\\
	\hline
\end{longtable}	
\end{center}

Los campos propios de la aplicación de programación por restricciones describen algunos aspectos de la ejecución usando programación por restricciones.
\newpage
\begin{center}
\begin{longtable}{|p{7cm}|p{9cm}|}
	\caption{Campos de información de ejecución usando programación por restricciones.}\\
	\hline
	\cellcolor[gray]{0.9} \textbf{Campo} & \cellcolor[gray]{0.9}\textbf{Descripción} \\
	\hline
	considerTop & Indica si se considera la restricción del tope de canales en la ejecución.\\
	\hline
	staticAssignation & Especifica si la asignación de canales de los operadores que requieren asignación se mantiene.\\
	\hline
	considerSeparation & Muestra si en la ejecución se ha tomado en cuenta la separación de canales de operadores diferentes.\\
	\hline
	spacesCreated & Número de espacios computacionales creados en la ejecución.\\
	\hline
	spacesSucceeded & Número de espacios computacionales exitosos en la ejecución.\\
	\hline
	FDVariables & Número de variables de estado finito creadas en la ejecución.\\
	\hline
	propagators & Número de propagadores creados durante la ejecución.\\
	\hline
\end{longtable}	
\end{center}

\subsection{Campos de una solución}

En el reporte de una solución tiene dos campos de información; el primero que son los costos de la solución y el segundo que es el reporte de la solución.

\subsubsection{Campos de los costos de una solución}

\begin{center}
\begin{longtable}{|p{7cm}|p{9cm}|}
	\caption{Campos de costos de una solución.}\\
	\hline
	\cellcolor[gray]{0.9} \textbf{Campo} & \cellcolor[gray]{0.9}\textbf{Descripción} \\
	\hline
	blocksNumber & Número de bloques existentes en la solución.\\
	\hline
	difChannelNumberMaxBlockFree & Diferencia entre el tamaño del bloque libre más grande y el número de canales.\\
	\hline
	channelNumberUseless & Número de canales inutilizados por separación.\\
	\hline
	totalCost & Costo total de la solución, que consiste en sumar los anteriores costos multiplicados cada uno por un peso especificado por el usuario.\\
	\hline
	violations & Número de violaciones a las restricciones. \textbf{Sólo aplica en algoritmo genético}.\\
	\hline
\end{longtable}	
\end{center}

\subsubsection{Campos de reporte de una solución}

\begin{center}
\begin{longtable}{|p{7cm}|p{9cm}|}
	\caption{Campos de información de un reporte de una solución específica.}\\
	\hline
	\cellcolor[gray]{0.9} \textbf{Campo} & \cellcolor[gray]{0.9}\textbf{Descripción} \\
	\hline
	operator & Indica cual es la asignación del operador.\\
	\hline
	channels & Asignación final para un operador dado.\\
	\hline
\end{longtable}	
\end{center}

