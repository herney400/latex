\section{Formato de entradas} 

A continuación se va definir el formato de las entradas para los aplicativos del proyecto, un ejemplo de entrada puede ser consultado en el anexo entradas de ejemplo ubicado en la página \pageref{ejemploEntrada}.

\subsection{Formato general}
Para las entradas se utiliza el formato XML usando la especificación \textit{dict} de XCSP\cite{XCSP}.
\\\\
Para definir una entrada de acuerdo al formato \textit{dict}, se debe asociar un orden convencional a un diccionario. Este orden convencional especifica un orden de llaves que pueden ser usadas para acortar notaciones. Mientras el grupo de llaves pueda ser conocido desde el contexto, los valores de cada llave pueden ser escritos en una notación abreviada, listando los valores de cada llave que son conocidas en el diccionario.
\\\\
Por ejemplo, las coordenadas de un punto pueden ser representadas por un diccionario que contiene dos llaves $x$ e $y$ con valores (2,5) respectivamente.

\lstset{frameround=fttt}
\begin{lstlisting}[frame=trBL, language=XML]
<dict>
<entry key="x"><i>2</i></entry>
<entry key="y"><i>5</i></entry>
</dict>
\end{lstlisting}
\subsection{Especificación de llaves}
Para el proyecto se han definido las siguientes llaves:

\begin{center}
\begin{longtable}{|p{7cm}|p{9cm}|}
	\caption{Estructura de llaves en las entradas}\\
	\hline
	\cellcolor[gray]{0.9} \textbf{Llave} & \cellcolor[gray]{0.9}\textbf{Descripción} \\
	\hline
	GeograficAssignationType & Tipo de asignación geográfica, 0 indica nacional, 1 territorial o regional, 2 departamental y 3 municipal.\\
	\hline
	GeograficAssignationID & ID o llave de la entidad territorial, en el caso de asignación nacional este valor es 0.\\
	\hline
	FrequencyBand & ID o llave banda de frecuencia de trabajo. \\
	\hline
	FrequencyRank & ID o llave del rango de frecuencia o banda específica de trabajo.\\
	\hline
	NumberChannels & Número de canales que tiene la canalización del rango de frecuencia.\\
	\hline
	NumberPresentOperators & Número de operadores que tienen asignación actualmente en el rango de frecuencia.\\
	\hline
	NumberOfOperatorWithRequirements & Número de operadores que requieren asignación.\\
	\hline
	ChannelSeparation & Separación mínima de canales entre dos operadores diferentes.\\
	\hline
	PresentOperators & Lista de operadores que tienen asignación en la banda actualmente.\\
	\hline
	OperatorsWithRequeriments & Lista de operadores que requieren asignación.\\
	\hline
	Requeriments & Lista de los requerimientos en canales que tienen los operadores que requieren asignación.\\
	\hline
	MaxAssignationsSubDivision & Lista que indica el máximo número de canales que tiene asignado en una división del área geográfica cada operador que solicita asignación.\\
	\hline
	ChannelAssignInDivisions & Indica que canales están asignados en las divisiones del área geográfica.\\
	\hline
	ReservedChannels & Indica que canales están reservados en las divisiones del área geográfica.\\
	\hline
	DisabledChannels & Indica que canales están deshabilitados en las divisiones del área geográfica.\\
	\hline
	ChannelAssignation & Matriz que especifica la asignación actual para los operadores presentes.\\
	\hline
	MaxChannelAssignationByOperator & Indica el máximo número de canales que puede tener asignado un operador en la banda.\\
	\hline
\end{longtable}	
\end{center}


